\section{Design of the Chat Application}
The development of our chat application follows a structured and iterative
approach, ensuring flexibility and continuous improvements throughout the
project. We began with an idea-storming phase, where we explored various
features, user needs, and technical requirements. This brainstorming session
helped us lay a strong foundation for the application's design and
functionality.

To maintain a clear and systematic development process, we employed Unified
Modeling Language (UML) to create both a scenario model and a design model. The
scenario model helped us visualize different user interactions and workflows
within the application, while the design model provided a structured blueprint
for the system architecture. These models serve as a guiding framework,
ensuring that all components of the application are well-structured and
logically connected.

For the development, we chose Java as the primary programming language due to
its robustness, cross-platform compatibility, and extensive support for
network-based applications. Additionally, we utilized SQL for database
management to handle user data, messages, and other relevant information
efficiently. This combination of technologies ensures that our chat application
is both scalable and secure.

\subsection{Overview of Software Architecture}
When it comes to designing a chat application one of the important things to
consider is identifying the needs, delegating responsibilities and ensuring data
integrity along the process of transferring information between sever and
client. For instance how are we going to make sure a client cannot send
messages to chat room they are not a member to? How are we going to disallow
tampering with user information? Who is responsible for informing the user of
new messages arriving at certain chat room? These are questions that must be
addressed as early as possible during the design phase to allow seamless
workflow during the implementation phase. 

The first design decision we had to make is to determine how we can store data
including user information, chat rooms and chat log. The idea we agreed upon 
eventually was use a SQL database to store all data. But the question was how are
we going to send information to the database. Are we going to allow the user application
to interact immediately with a database or is there going to be an application on 
the server which hosts the database that define the protocols of interaction and 
data transfer? There were two main problems with allowing immediate

\subsection{Server-Side Design}
\subsubsection{Data Storage and Management}
\subsubsection{Server TCP Communication}
\subsection{Client-Side Design}
\subsubsection{Client Authentication}
\subsubsection{Client TCP Communication}
\subsubsection{Client Interaction and User Interface}
\subsection{Class Design with CRC Cards}